\documentclass[12pt]{article}
\title{\textbf{Soluci\'on Problema - Vol. IX-3}}
\author{Ricardo Ruiz Fern\'andez de Alba 2 Bach - IES Albor\'an}
\date{24/09/2016}
\usepackage{lmodern}
\usepackage[T1]{fontenc}
\usepackage[spanish,activeacute]{babel}
\usepackage{mathtools}
\usepackage{amsfonts}

\begin{document}

\maketitle

\section*{Determina todos los conjuntos de n'umeros naturales consecutivos cuya suma vale 91}

Necesitamos hallar todos los subconjuntos de $\mathbb{N}$ de n'umeros consecutivos que sumen 91, es decir, que tengan la forma:

\begin{align*}
	\{m, m+1, m+2, ..., n \} \subset \mathbb{N} \\
\end{align*}

Y la suma de todos sus elementos ser'a:
\begin{align*}
	\sum_{k=m}^nk = 91
\end{align*}

Si pensamos que, $$3+4+5= (1+2+3+4+5) - (1+2)$$ y generalizamos, podemos expresar este conjunto como la diferencia entre dos \emph{segmentos de sucesi'on natural.}
\begin{align*}
\{m, m+1, m+2, \ldots, n\} = |1, n| \setminus |1, m-1|\\
\end{align*}
As'i pues, la suma ser'a
$$
\sum_{k=m}^nk = \sum_{k=1}^nk- \sum_{k=1}^{m-1}k = 91
$$
Y, considerando la conocida f'ormula
$$
\sum_{k=1}^nk=\frac{n(n+1)}{2}
$$
que queda demostrada en el anexo, podemos expresar la suma finalmente como

$$
\sum_{k=m}^nk = \sum_{k=1}^nk- \sum_{k=1}^{m-1}k =\frac{n(n+1)}{2} - \frac{(m-1)(m-1+1)}{2} =\frac{n(n+1) - m(m-1)}{2} = 91
$$

Como buscamos los $(n,m)\in\mathbb{N}$ que cumplan esta ecuaci'on, se tratar'ia de una \emph{Ecuaci'on "diof'antica" (limitada a $\mathbb{N}$ en lugar de $\mathbb{Z}$})

Simplificando,

\begin{align*}
\frac{n(n+1) - m(m-1)}{2} = 91 \\
n(n+1) - m(m-1) = 182 \\
n^2+n-m^2+m = 182 \\
n^2-m^2+n+m = 182 \\
(n+m)(n-m)+(n+m) = 182 \\
\end{align*}

Y finalmente, sacando factor com'un, nos queda:
$$
(n+m)(n-m+1) = 182
$$
Si consideramos a $182$ como producto de dos factores naturales $b$ y $c$, podemos convertir la ecuaci'on en un \emph{Sistema lineal de dos ecuaciones con dos inc'ognitas.}
\begin{align*}
(n+m)(n-m+1) = 182 = bc \\
\left\{ 
\begin{array}{c}
n + m= b \\ 
n-m +1 = c
\end{array}
\right. 
\end{align*}
 
Resolviendo el sistema por reducci'on, se obtiene:
$$
   n = \frac{b+c-1}{2}, m = \frac{b-c+1}{2}
$$
Adem'as, se tiene que cumplir que $n > 0$, $m > 0$ y que $n > m$

\begin{align*}
n = \frac{b+c-1}{2} > 0; b > 1-c \\
m = \frac{b-c+1}{2} > 0; b > c -1 \\
\frac{b+c-1}{2} > \frac{b-c+1}{2} \\
2c > 2; c > 1; c - 1 > 0
\end{align*}

De tal manera que la condici'on resumen es
$$
b > c -1 > 0
$$

Siendo $182 = 2\cdot7\cdot13$,  las 'unicas parejas de factores que cumplen esta condici'on son:

\begin{align*}
(I)  \quad b= 2\cdot7 = 14, c = 13 \\
(II) \quad b = 2\cdot13 = 26, c = 7 \\
(III)\quad b = 7\cdot13 = 91, c= 2
\end{align*}


Y los correspondientes $n$ y $m$ ser'ian:


$(I)$

\begin{align*}
n=\frac{14+13-1}{2}=13\\
m = \frac{14-13+1}{2} = 1\\
\end{align*}


$(II)$

\begin{align*}
n = \frac{26+7-1}{2} = 16\\
m = \frac{26-7+1}{2} = 10
\end{align*}


$(III)$
\begin{align*}
n = \frac{91+2-1}{2} = 46\\
m = \frac{91-2+1}{2} = 45
\end{align*}

Por tanto, los subconjuntos de n'umeros consecutivos en $\mathbb{N}$ que suman $91$ son:

\begin{align*}
\{1,2,3,4,5,6,7,8,9,10,11,12,13\}\\
\{10, 11, 12, 13, 14, 15,16\}\\
\{45, 46\}
\end{align*}

\section*{Anexo: Demostraci'on de la f'ormula utilizada}
Se va a demostrar por inducci'on la siguiente proposici'on.

$$
P(n) = \sum_{k=1}^nk = \frac{n(n+1)}{2} 
$$
1) $P(1)$ es cierta
$$
     1 = \frac{1(1+1)}{2}
$$
2) Suponiendo $P(n)$ cierta, demostramos $P(n+1)$

\begin{align*}
	\sum_{k=1}^{n+1}k = \frac{(n+1)(n+2)}{2} = \\
	=\frac{n^2+2n+n+2}{2} = \frac{n^2+n}{2} + \frac{2n+2}{2} \\
	= \frac{n(n+1)}{2} + n+1 = \\
	 \text{Y, por hip'otesis de inducci'on,} \\
	 = \left(\sum_{k=1}^nk\right)+(n+1) =
	  \sum_{k=1}^{n+1}k
\end{align*}


\end{document}